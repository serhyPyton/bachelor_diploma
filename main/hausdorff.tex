\chapter{Відстань Хаусдорфа}
\section{Теоретичні основи}
Метричний простір \cite{DorogovtsevMA}~---~це пара $ \left( S, d \right) $,
яка складається з множини $S$ і метрики
$d \; : \; S \times S \rightarrow \mathbb{R}$,
тобто для будь-яких $x, y, z \in S$ виконується
\begin{enumerate}
  \item $d \left( x, y \right) \geq 0$;
  \item $d \left( x, y \right) = 0 \Longleftrightarrow x = y$~---~аксиома тотожності;
  \item $d \left( x, y \right) = d \left( y, x \right) $~---~аксиома симетрії;
  \item $d \left( x, z \right) \leq
    d \left( x, y \right) + d \left( y, z \right) $~---~нерівність трикутника.
\end{enumerate}

Метричний простір $ \left( S, d \right) $ називається сепарабельним,
якщо існує не більше ніж зліченна множина $ \Gamma \subset S$ така, що
\begin{equation*}
  \left( \forall x \in S \right) \,
  \left( \forall \varepsilon > 0 \right) \,
  \left( \exists y \in \Gamma \right) \, : \qquad
  d \left( x, y \right) < \varepsilon.
\end{equation*}

Метричний простір $ \left( S, d \right) $ називається повним,
якщо в ньому збігається будь-яка фундаментальна послідовність.
Прикладом повного сепарабельного метричного простору є $ \mathbb{R}^n$
з евклідовою відстанню.

\section{Відстань Хаусдорфа}

Відстань Хаусдорфа визначається на
множині всіх непустих замкнених підмножин простору $ \mathbb{R}^n$.

Нехай $S$ і $T$~---~непусті замкнені підмножини $ \mathbb{R}^n$.
Відстань Хаусдорфа між $S$ і $T$ визначається як
\begin{equation}\label{eq:hausdorff:distance}
  H \left( S, T \right) =
  \inf \left\{
    \varepsilon \geq 0 \; \middle| \;
    S \subset T + \varepsilon, \, T \subset S + \varepsilon
  \right\},
\end{equation}
де $S + \varepsilon $~---~об'єднання замкнених шарів
з радіусом $ \varepsilon $ і центром в точці $x \in S$
\begin{equation*}\label{eq:set:expansion}
  S + \varepsilon =
  \bigcup \limits_{x \in S}
    \left\{ \overline{B}_{ \varepsilon } \left( x \right) \right\}.
\end{equation*}

Перевіримо аксіоми метрики для відстані Хаусдорфа $H \left( S, T \right) $,
яка задана формулою \eqref{eq:hausdorff:distance}.

\begin{enumerate}
  \item $H \left( S, T \right) \geq 0$.
  Це випливає з означення \eqref{eq:hausdorff:distance},
  бо точна нижня межа величини $ \varepsilon \geq 0$ невід'ємна.
  \item $H \left( S, T \right) = 0$ тоді та тільки тоді, коли $S = T$.
  Остання рівність рівносильна двом умовам: $S \subset T$ та $T \subset S$.
  Це можна записати через елементи множин:
  якщо $x \in S$, то $x \in T$, та якщо $x \in T$, то $x \in S$.

  Нехай $x \in S$ і $ \forall \varepsilon > 0$ виконується
  $T + \varepsilon \supset S$, тобто $x \in T + \varepsilon $.
  Використаємо означення \eqref{eq:set:expansion}
  \begin{equation*}
    x \in
    \bigcup \limits_{y \in T} \overline{B}_{ \varepsilon } \left( y \right).
  \end{equation*}
  Якщо $x$ належить об'єднанню множин,
  то $x$ належить хоча б однієї з цих множин.
  Отже, знайдеться такий $y_{ \varepsilon } \in T$,
  що $x \in \overline{B}_{ \varepsilon } \left( y_{ \varepsilon } \right) $,
  тобто $d \left( y_{ \varepsilon }, x \right) \leq \varepsilon $.
  Це виконується для будь-якого $ \varepsilon \geq 0$, отже,
  $x$ або лежить в $T$, або є його граничною точкою.
  Але $T$~---~замкнена множина, звідки випливає, що $x \in T$.

  Друга частина доводиться аналогічно.

  З іншого боку, якщо $S = T$, то $S \subset T$ та $T \subset S$,
  отже, $ \varepsilon = 0$ та $H \left( S, T \right) = 0$.
  \item $H \left( S, T \right) = H \left( T, S \right) $
  випливає з симетричності означення відстані Хаусдорфа.
  \item $H \left( S, T \right) \leq
    H \left( S, T \right) + H \left( T, G \right) $
  для будь-яких замкнених множин $S, T, G$ з $ \mathbb{R}^n$.
  Треба перевірити, чи виконується наступне
  \begin{equation*}\label{eq:triangle}
    \left. \begin{aligned}
      \varepsilon_{S, G} \geq H \left( S, G \right), \\
      \varepsilon_{G, T} \geq H \left( G, T \right)
    \end{aligned} \right \rbrace \overset{?}{\Rightarrow}
    S \subset T + \varepsilon_{G, T} + \varepsilon_{G, S}.
  \end{equation*}

  Виконуємо ті ж дії, що й при перевірці другої умови.
  Для першого рядка системи отримуємо, що з того,
  що $x \in S$ і $G + \varepsilon_{S, G} \supset S$, випливає, що $x \in G$.
  Використовуючи умову з другого рядка, отримуємо,
  що при цьому $T + \varepsilon_{G, T} \supset G$,
  тобто $x \in T + \varepsilon_{G, T}$, отже,
  при $ \varepsilon_{S, G} \geq 0$ вконується й
  $x \in T + \varepsilon_{G, T} + \varepsilon_{S, G}$.
  Згадуючи, що з самого початку $x$ належав множині $S$, бачимо,
  що наслідок \eqref{eq:triangle} виконується,
  тобто нерівність трикутника справедлива.
\end{enumerate}
Аксиоми метрики виконуються, отже,
відстань Хаусдорфа~---~метрика на замкнених множинах з $ \mathbb{R}^n$.

\subsection{Приклад 1}

Знайдемо відстань Хаусдорфа між двома еліпсами
(рис.~\ref{fig:hausdorff:example}) \cite{crownover}
\begin{equation*}
  \begin{gathered}
    S \, : \, \frac{x^2}{4} + 4y^2 = 1, \, \\
    T \, : \, 4 \left( x - 2 \right)^2 + \frac{y^2}{4} = 1.
  \end{gathered}
\end{equation*}

\begin{figure}[h]
  \centering
  \includestandalone[mode=buildnew]{../tikz/ellipsesHausdorff}
  \caption{Еліпси, між якими шукаємо відстань Хаусдорфа}
  \label{fig:hausdorff:example}
\end{figure}

Пунктиром нарисовані еліпси $S + \varepsilon $ і $T + \varepsilon $ такі,
щобб виконувалось \eqref{eq:hausdorff:distance}.
У даному випадку $ \varepsilon $~---~це відстань між точками $A$ та $B$,
яка дорівнює $ \varepsilon = 1.5 - \left( -2 \right) = 3.5$.
Тому $H \left( S, D \right) = 3.5$.

\subsection{Приклад 2}

Нехай $ \left( X, d \right) $~---~метричний простір і $A \subset X$.
Нехай $ \varepsilon > 0$ задано.
Множина $C$~---~$ \varepsilon $-сітка для множини $A$,
якщо для кожного $x \in A$ знайдеться такий $y \in C$,
що $d \left( x, y \right) < \varepsilon $ або
\begin{equation*}
  \bigcup \limits_{y \in C} B \left( y, \varepsilon \right) \supset A.
\end{equation*}
Тоді відстань Хаусдорфа між множиною $A$ й $ \varepsilon $-сіткою $C$ на ній
$H \left( C, A \right) = \varepsilon $.

Покажемо це.
Знайдемо відстань Хаусдорфа
\begin{equation*}
  H \left( C, A \right) =
  \inf \left\{
    \varepsilon \geq 0 \; \middle| \;
    C + \varepsilon \supset A, \, A + \varepsilon \supset C
  \right\}.
\end{equation*}
За означенням
\begin{equation*}
  C + \varepsilon =
  \bigcup \limits_{y \in C} \overline{B} \left( y, \varepsilon \right) \supset
  \bigcup \limits_{y \in C} B \left( y, \varepsilon \right) \supset
  A.
\end{equation*}
З іншого боку,
\begin{equation*}
  A + \varepsilon =
  \bigcup \limits_{x \in A} \overline{B} \left( x, \varepsilon \right) \supset
  A \supset C.
\end{equation*}
Отримали, що $C + \varepsilon \supset A$ й $A + \varepsilon \supset C$,
тобто $H \left( C, A \right) = \varepsilon$.

Таким чином, на неперервній множині можна задавати
$\varepsilon$-сітку та отримувати точкову множину,
яка відрізняється від початкової не меньше ніж на $\varepsilon$.
