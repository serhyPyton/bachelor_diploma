\chapter*{Вступ}
\addcontentsline{toc}{chapter}{Вступ}

Значною мірою робота завдячує Андрію Анатолійовичу Дороговцеву~---~професору,
доктору фізико-математичних наук,
завідувачу відділу теорії випадкових процесів Інституту математики НАН України.

\textbf{Актуальність роботи.}
Оцінка положення камери по хмарам точок (або точковим множинам)
лежить в основі сканування об'єктів за допомогою 3D сканера,
одночасній локалізації і картографування.
Для розв'язання цих задач використовується
ітеративний алгоритм найближчих точок і його модифікації.
У зв'язку з розвитком та компактизацією обчислювальної техніки
з'явилась можливість реалізовувати алгоритм на маленьких комп'ютерах (наприклад,
бортові комп'ютери дронів).
Оскільки потрібно, щоб такі пристрої працювали надійно,
треба ретельно вивчити властивості алгоритму, що використовується.

\textit{Об'єкт дослідження} ---
методи оцінки параметрів камери.

\textit{Предмет дослідження} ---
алгоритм співставлення точкових множин.

\textbf{Мета дослідження.}
Аналіз алгоритму співставлення двох точкових множин
та отриманих за його допомогою оцінок невідомих параметрів.

Завдання наступні:
\begin{enumerate}
  \item
    застосувати метод найменших квадратів для розв'язання задачі;
  \item
    ознайомитися з ітеративним алгоритмом найближчих точок,
    що використовується для співставлення двох точкових множин;
  \item
    перевірити однозначність оцінки, яка є результатом роботи алгоритму;
  \item
    дослідити алгоритм на збіжність;
  \item
    розробити програмну реалізацію алгоритму.
\end{enumerate}

\textbf{Практичне значення одержаних результатів.}

Ітеративний алгоритм найближчих точок можна використовувати для відновлення
двовимірних або тривимірних поверхонь, отриманих за допомогою 3D сканера.
Було з'ясовано, що алгоритм визначений однозначно, тобто за однакових вхідних
даних він дає один і той самий результат.
Це додає зручності для подальшого дослідження,
а також спрощує реалізацію алгоритму, бо не виникає необхідності вибору
оптимального розв'язку в умовах його неоднозначності.

\textbf{Публікації.}

XVI Всеукраїнська науково-практична конференція студентів,
аспірантів та молодих вчених <<Теоретичні і прикладні проблеми фізики,
математики та інформатики>>.
