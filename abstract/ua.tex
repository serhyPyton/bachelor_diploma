\chapter*{Реферат}

Дипломна робота містить \pageref{LastPage} сторінки,
\TotalValue{totalfigures} ілюстрацій, 1 додаток і
\total{citenum} джерел літератури.

Однією з актуальних проблем компьютерного зору є оцінка положення камери по
точковим множинам.
Вона лежить в основі сканування об'єктів за допомогою 3D сканера,
одночасній локалізації і картографування.

Об'єктом дослідження є методи оцінки параметрів камери.

Предметом дослідження є алгоритм співставлення точкових множин.

Метою даної роботи є дослідження алгоритму співставлення
двох точкових множин як методу оцінки положення камери та отриманих за його
допомогою оцінок невідомих параметрів.

Для досягнення мети будо використано
\begin{itemize}
  \item метод найменших квадратів, щоб знайти ефективні оцінки шуканих
  параметрів;
  \item сингулярний розклад матриці, щоб забезпечити ортогональність оцінки
  матриці повороту;
  \item відомості з теорії ймовірностей для доведення однозначності оцінок та
  знаходження їх розподілу.
\end{itemize}

\MakeUppercase{ітеративний алгоритм найближчих точок, метод найменших квадратів,
сингулярний розклад матриці, тривимірні точкові множини,
одначасна локалізація і картографування}
