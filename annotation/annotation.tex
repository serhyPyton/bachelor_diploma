\input{../common/head_ua.tex}

\begin{document}
\pagenumbering{gobble}
\chapter*{Анотація}

Iдентифiкацiя випадкових вiдображень точкових множин в
скiнченновимiрних просторах. / Лавягіна О.О.
Звіт з переддипломної практики за спеціальністю $6.040301$
<<Прикладна математика>>, Київ, 2018.

Місце проходження практики: НТУУ <<КПІ ім. Ігоря Сікорського>>,
Фізико-технічний інститут, кафедра інформаційної безпеки.

Керівник: д.ф.-м.н. Рябов Георгій Валентинович.

Метою роботи є аналiз алгоритму спiвставлення двох точкових
множин та отриманих за його допомогою оцiнок невiдомих параметрiв.

Методом дослідження було опрацювання літератури за даною темою,
теоретична та практична перевірка роботи ітеративного алгоритму найближчих
точок та опрацювання отриманих результатів.

Ключові слова: МЕТОД НАЙМЕНШИХ КВАДРАТІВ, СИНГУЛЯРНИЙ РОЗКЛАД МАТРИЦІ,
ТРИВИМІРНІ ТОЧКОВІ МНОЖИНИ, СКАНУВАННЯ ТРИВИМІРНИХ ОБ'ЄКТІВ.
\end{document}
