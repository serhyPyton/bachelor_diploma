\chapter*{Висновки}
\addcontentsline{toc}{chapter}{Висновки}

В результаті виконання роботи було досліджено задачу співставлення двох
точкових множин та ітеративний алгоритм найближчих точок, що її розв'язує.
Для розв'язання поставленої задачі було використано метод найменших квадратів,
сингулярне розкладення та відомості з теорії ймовірностей.
З'ясовано, що звичайний метод найменших квадратів не дає оптимального
розв'язку в даному випадку,
адже на шукані параметри накладені нелінійні обмеження.

Доведено, що ітеративний алгоритм найближчих точок дає однозначну оцінку шуканих
параметрів.
Знайдено розподіл оцінки матриці повороту та доведено,
що алгоритм завжди зупиняється за скінченну кількість ітерацій.

Було реалізовано демонстративне програмне забезпечення,
що за допомогою ітеративного алгоритму найближчих точок точно оцінює шукані
параметри.

На даний момент поставлена задача правильно розв'язується тільки повним
перебором, який неможливо практично застосовувати через час його
роботи для множин, що складаються з великої кількості точок.
